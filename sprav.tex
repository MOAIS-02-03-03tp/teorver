\documentclass[12pt]{article}
\usepackage[a4paper, total={7in,10in}]{geometry}
\usepackage{polyglossia}
\usepackage{ragged2e}
\usepackage{amsmath}
\usepackage{amssymb}
\usepackage{microtype}
\usepackage{graphicx}
\let\ORIincludegraphics\includegraphics
\renewcommand{\includegraphics}[2][]{\ORIincludegraphics[scale=0.65,#1]{#2}}
\usepackage{changepage}
\usepackage{hyperref}
\usepackage{cancel}
\graphicspath{{./images/}}
\setmainlanguage{russian}
\setotherlanguage{english}
\newfontfamily\russianfont[Script=Cyrillic]{Times New Roman}
\newfontfamily\englishfont{Times New Roman}
\setlength{\parindent}{0em}
\setlength{\parskip}{6pt}

\def\posl#1#2{\{#1_{#2}\}}
\DeclareMathOperator*{\sh-like}{\sinh-like}
\DeclareMathOperator*{\ch-like}{\cosh-like}
\DeclareMathOperator*{\th-like}{\tanh-like}
\DeclareMathOperator*{\cth-like}{\coth-like}
\DeclareMathOperator*{\tg-like}{\tan-like}
\DeclareMathOperator*{\ctg-like}{\cot-like}
\DeclareMathOperator*{\arctg-like}{\arctan-like}
\DeclareMathOperator*{\arcctg-like}{\arctan-like}

\begin{document}
\section{Вероятность}
    $\varOmega$ - эл. исходы $\omega \in \varOmega$, A $\sqsubset \varOmega$ сл. соб.\\
    Опр: A алгебра: $A \in A; \varOmega \in A; A \in A,B \in A \Rightarrow A \lor B, A \land B, \overset{-}{A} \in A$ \\%вторая А везде выебывается, хз как в латексе ее сделать%
    $\sigma$- алгебра: $\forall \{A_1,\dots,A_n,\dots\}\sqsubset A \Rightarrow U_1^{\infty} A_n \in A$\\
    Опр: $\Omega, A$ вер-ть: P $A \Rightarrow R$\\
    \begin{enumerate}
        \item $P(A) \leq o, \forall A \in A$
        \item $P(\varOmega)=1$
        \item $\forall \{A_n\}\sqsubset A,A_i \land A_j = \not o \Rightarrow P(U_1^{\infty}A_n)=\sum_{1}^{\infty}
        P(A_n) \; \sigma$ - индуктивности
    \end{enumerate}
    A+$\overset{-}{A}=\varOmega$\\
    $P(A+\overset{-}{A})=P(A)+P(\overset{-}{A})=1 \Rightarrow P(A)=1-P(\overset{-}{A})$\\
    $P(A+B)=P(A)+P(B),AB=\not o$
\end{document}